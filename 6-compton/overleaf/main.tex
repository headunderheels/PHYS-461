\documentclass[a4paper,12pt,english]{all-in-one} %% TWOSIDE
\usepackage{amsmath} 
\usepackage{newtxmath}
\usepackage{multicol}
\usepackage{lipsum}
\usepackage{subfig}
\usepackage{listings}
\usepackage[hidelinks]{hyperref}
\renewcommand\theequation{\arabic{equation}}
\newcommand\tab[1][1cm]{\hspace*{#1}} 

\doctitle{Modern Physics Laboratory }
\docsubtitle{Compton effect} % Experiment name here

\makeatletter
\title{{\large\textit{Modern Physics Laboratory | PHYS-461}}\\[0.5cm]{\Huge\color{gray}\textsc{\@docsubtitle}}}
\makeatother

\author{\textbf{Cordney Nash}  \and Micah Hillman  }
\date{November 13, 2024}
\footext{}



\begin{document}

\begin{titlepage}
\maketitle\vfill
\end{titlepage}
\newpage


\section*{Introduction}
{
In the Compton effect, an incoming photon collides elastically with an electron, resulting in the scattering of the photon with altered momentum and direction. The theory behind Compton scattering combines principles of relativistic and quantum mechanics, as it deals with massless photons and involves substantial energy transfers relative to the electron’s rest energy. While the Compton effect applies to photons of various energies, including visible light, this experiment focuses on higher-energy photons, specifically X-rays and gamma rays, to observe and analyze the effect more clearly. In the end we wish to determine the rest energy $E_0$ of the electron.
}

\begin{multicols}{2}

\section*{Theory \& Procedure}
{
The first step in starting the experiment is to open the UCS-30 MCA software and correctly calibrate the system. This is done with a Cesium-137 ($Cs-137$) and Sodium-22 ($Na-22$) radiation source. To effectively calibrate we need three peaks: the first two, 662 $keV$ and 32 $keV$, are found from $Cs-137$ and the third point at 511 $keV$ is found from $Na-22$. This calibration is done each day we decided to take measurements. We must also be sure that the High Voltage is set to 850, the Coarse Gain is set to 16 and, Fine Gain is shifted to 1.17.

Before we can start the second step we must know that we are to measure from 10$^\circ$-110$^\circ$ in increments of 10$^\circ$. For each degree we must measure for 60 seconds. So for example, at 70$^\circ$ we must measure for 4,200 seconds. We must also know to take data with the scatterer on and removed, this will count as the spectrum and background respectively. The difference in these histograms are due purely to Compton scattering, meaning the noise is subtracted. The second major step is to simply measure, starting at 10$^\circ$. Once each measurement is done we are sure to save our data.  
}

\section*{Analysis \& Results}
{

}
\end{multicols}

\section*{Summary}
{

}



\end{document}