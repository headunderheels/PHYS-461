\documentclass[12pt]{report}

\usepackage{newpxtext} % Palatino text font
\usepackage{newpxmath} % Palatino math font
\usepackage{microtype} % Better text alignment
\usepackage[normalem]{ulem} % Underline
\usepackage{xparse} % Advanced command definitions
\usepackage{contour} % Outline text
\usepackage{lipsum} % Dummy text
\usepackage{mhchem}
\usepackage{siunitx}
\usepackage{graphicx}
\usepackage{subcaption}
\usepackage{multicol}
\usepackage[style=numeric, backend=biber]{biblatex} %Imports biblatex package

\graphicspath{{./images/}}

\contourlength{0.3pt} % Default contour length

\NewDocumentCommand{\ul}{O{3pt} O{0.55pt} O{1.75pt} m}{%
  \begingroup%
  \renewcommand\ULdepth{#1}%
  \renewcommand\ULthickness{#2}%
  \contourlength{#3}%
  \uline{\phantom{#4}}\llap{\contour{white}{#4}}%
  \endgroup%
}

\DeclareCaptionFormat{custom}
{%
    \textbf{#1.} #3
}
\captionsetup{format=custom}

\newcommand{\goldenRatio}{1.6180}
\newcommand{\inverseGoldenRatio}{0.6180}
\addbibresource{bib.bib}


\title{Temperature Effects on Phosphor Fluorescence Lifetime \\\vspace{0.5cm}
\large{PHYS 461}}
\date{September 16, 2024}
\author{\ul{Micah Hillman} \and Cordney Nash}

\begin{document}
  \maketitle

  \section*{Introduction}{
    Europium-doped phosphor compounds can exhibit temperature-dependent fluorescence lifetimes for some emission lines. In europium-doped lanthanum oxysulfide (\ce{La2O2S{:}Eu}), the variable overlap of a charge-transfer (CT) state with the \ce{^{5}D_{i}} energy levels leads to an increased availability of non-radiative de-excitation pathways as temperature is increased. For lower temperatures, the CT state becomes less available, and radiative emission dominates, leading to longer fluorescence lifetimes \cite{parks}. We measured fluorescence lifetimes for a sample of \ce{La2O2S{:}Eu} between \SI{-10}{\degreeCelsius} and \SI{100}{\degreeCelsius}, and observed an exponential decrease in decay lifetimes for increasing temperatures.
  }

  \section*{Methods}{

  \begin{figure}[h]
    \centering
    \includegraphics[width=0.7\linewidth]{experiment_diagram}
    \caption{A diagrammatic representation of the experimental setup.}\label{fig:setup}
  \end{figure}

    To modulate its temperature, the phosphor sample was mounted on a Peltier device attached to a manually-variable current source (\textbf{Figure~\ref*{fig:setup},~\(\bf{D}\)}). Focused light from a \SI{365}{\nano \meter} pulsing laser diode (\(\bf{A}\)) was reflected (\(\bf{B}\)) and focused (\(\bf{C}\)) on the surface of the sample, causing fluorescence at the \SI{514}{\nm}, \ce{^{5}D2} emission line (among others). Fluoresced light was then band-passed (\(\bf{E}\)) and focused (\(\bf{F}\)) into a photomultiplier tube (PMT) (\(\bf{G}\)). The PMT-amplified fluorescence response signal was then passed with the original impulse signal to be overlayed on a digital oscilloscope for data collection.
    
    After setting the pulse width of the laser diode to approximately \SI{1}{\micro\second}, we began varying the current supplied to the Peltier device to set the temperature at approximate steps of \SI{10}{\degreeCelsius} ranging from \SI{-10}{\degreeCelsius} to \SI{100}{\degreeCelsius}. Three snapshots of oscilloscope data were collected at each increment, where the oscilloscope timing window was variably tuned to meet the following specifications:
      
      \begin{enumerate}
        \item maximize timing resolution by including as many non-zero response values as possible, and
        \item include information about the fluorescence response's offset prior to the laser impulse (for offset subtraction during analysis).
      \end{enumerate}
  }

  \section*{Results}

    Time series data including signal (LED) and response (PMT) voltages collected from the oscilloscope were used to create \texttt{pandas} data frames for visualization and analysis. There was a baseline offset for every measurement of the response voltage $V_{\mathrm{resp.}}$, which we removed by calculating the average of the initial response voltage values and subtracting this value from all $V_{\mathrm{resp.}}$ values globally. To perform a linear fit, we plotted $\ln(V_{\mathrm{resp.}})$ as a function of time (Figure\,\ref{fig:f1}). A line of best fit, along with its fitting uncertainty, was determined for each time series. These fitting parameters were then used to calculate a decay constant, $\tau(T)$, according to the following relation:
    \begin{equation}
        \tau = -\frac{1}{m}, \label{eq}
    \end{equation}
    where $m$ is the slope of the linear fit. Fitting errors were propagated using the python \texttt{uncertainties} package.

    \begin{figure}[tbh]
        \centering 
        \includegraphics[width=.7\linewidth]{timeseries-fit.png} 
        \caption{An example of fitted data. A signal from the LED (shown in gray) elicits a response from the phosphor (black). A linear function was fit to the response curve, and the decay constant $\tau$ was calculated using Equation\,\ref{eq}.} 
        \label{fig:f1} 
    \end{figure}

    \begin{figure}[tbh] 
        \centering 
        \includegraphics[width=.7\linewidth]{image.png} 
        \caption{Phosphor decay lifetime, $\tau$, as a function of phosphor temperature, $T$, between \SI{-10}{\degreeCelsius} and \SI{100}{\degreeCelsius}. All error bars are scaled by a factor of $e^{3}$ for visibility. Errors are also presented in tabular form in Table \ref{tab:sample}.} \label{fig:fig2}
    \end{figure}

At each temperature value, we averaged our three calculated time constants and found the standard error of the mean for each point, combining it in quadrature with the mean of the fitting uncertainties\footnote{Our error bars were quite small relative to the change in temperature, so they were artificially inflated when plotting.}. On a logarithmic axis, we plotted the decrease in fluorescence as a function of temperature, finding that the decay constant, $\tau$, decreased exponentially with temperature within this region (Figure\,\ref{fig:fig2}), indicating the increasing availability of non-fluorescent de-excitation pathways for the \ce{^{5}D_{i}} energy levels at higher temperatures.

\begin{table}[ht]
\centering
\begin{tabular}{rrr}
\hline
   Avg. Temp., $\overline{T}\ (\mathrm{^{\circ}C})$ &   Avg. dec. lifetime, $\overline{\tau}\ (\upmu \mathrm{s})$ &   Dec. lifetime unc., $\sigma_{\tau}$ \\
\hline
      -10.1    &           553    &                      1.51    \\
        0.0667 &           312    &                      0.641   \\
       10.0    &           195    &                      0.420   \\
       20.2    &           111    &                      0.219   \\
       30.1    &            66.7  &                      0.116   \\
       40      &            37.7  &                      0.0968  \\
       50      &            24.5  &                      0.0519  \\
       59.9    &            14.2  &                      0.0613  \\
       70.1    &             8.5  &                      0.0349  \\
       80.2    &             5.13 &                      0.0237  \\
       90      &             3.43 &                      0.014   \\
      100      &             2.59 &                      0.00578 \\
\hline
\end{tabular}
\caption{Table of values and associated errors.}
\label{tab:sample}
\end{table}

\printbibliography

\end{document}
