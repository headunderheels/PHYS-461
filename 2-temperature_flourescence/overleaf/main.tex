\documentclass[12pt]{article}

\usepackage{newpxtext} % Palatino text font
\usepackage{newpxmath} % Palatino math font
\usepackage{microtype} % Better text alignment
\usepackage[normalem]{ulem} % Underline
\usepackage{xparse} % Advanced command definitions
\usepackage{contour} % Outline text
\usepackage{lipsum} % Dummy text
\usepackage{mhchem}
\usepackage{siunitx}
\usepackage{graphicx}
\usepackage{subcaption}
\usepackage{multicol}
\usepackage[style=numeric, backend=biber]{biblatex} %Imports biblatex package

\graphicspath{{./images/}}

\contourlength{0.3pt} % Default contour length

\NewDocumentCommand{\ul}{O{3pt} O{0.55pt} O{1.75pt} m}{%
  \begingroup%
  \renewcommand\ULdepth{#1}%
  \renewcommand\ULthickness{#2}%
  \contourlength{#3}%
  \uline{\phantom{#4}}\llap{\contour{white}{#4}}%
  \endgroup%
}

\DeclareCaptionFormat{custom}
{%
    \textbf{#1.} #3
}
\captionsetup{format=custom}

\newcommand{\goldenRatio}{1.6180}
\newcommand{\inverseGoldenRatio}{0.6180}

\title{Temperature Effects on Phosphor Fluorescence Lifetime}
\date{September 16, 2024}
\author{\ul{Micah Hillman} \and Cordney Nash}

\begin{document}
  \maketitle

  \section*{Introduction}{
    Europium-doped phosphor compounds can exhibit temperature-dependent flourescence lifetimes for certain emission lines. In europium-doped lanthanum oxysulfide (\ce{La2O2S{:}Eu}), the variable overlap of a charge-transfer (CT) state with the \ce{^{5}D_{i}} energy levels leads to an increased availability of non-radiative de-excitation pathways as temperature is increased. For lower temperatures, the CT state becomes less available and radiative emission dominates, leading to longer fluorescence lifetimes. We measured fluorescence lifetimes for a sample of \ce{La2O2S{:}Eu} between \SI{-10}{\degreeCelsius} and \SI{100}{\degreeCelsius}, and observed a (linear/logarithmic) decrease in decay lifetimes for increasing temperatures.
  }

  \section*{Methods}{

  \begin{figure}[h]
    \centering
    \includegraphics[width=0.7\linewidth]{experiment_diagram}
    \caption{Caption formatting test.}
    \label{fig:setup}
  \end{figure}

    To modulate its temperature, the phosphor sample was mounted on a Peltier device attached to a manually-variable current source. Focused light from a pulsing laser diode shone on the surface of the sample, causing fluorescence at the \SI{514}{\nm}, \ce{^{5}D2} emission line (among others). Fluoresced light was then band-passed and focused into a photomultiplier tube (PMT). The PMT-amplified fluorescence response signal was then passed with the original impulse signal to be overlayed on a digital oscilloscope for data collection.
    
    After setting the pulse width of the laser diode to approximately \SI{1}{\micro\second}, we began varying the current supplied to the Peltier device to set the temperature at approximate steps of \SI{10}{\degreeCelsius} ranging from \SI{-10}{\degreeCelsius} to \SI{100}{\degreeCelsius}. Three snapshots of oscilloscope data were collected at each increment, where the oscilloscope timing window was variably tuned to meet the following specifications:
      
      \begin{enumerate}
        \item maximize timing resolution by including as many non-zero response values as possible, and
        \item include information about the fluorescence response's offset prior to the laser impulse (for offset subtraction during analysis).
      \end{enumerate}
    
    
    
  }

  \section*{Results}{
    \lipsum[1]
  }

\end{document}
