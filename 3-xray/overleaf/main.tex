\documentclass[a4paper,12pt,english]{all-in-one} %% TWOSIDE
\usepackage{ebgaramond}
\usepackage{multicol}
\usepackage{lipsum}


\doctitle{Modern Physics Laboratory }
\docsubtitle{X-ray Bragg Diffraction} % Experiment name here

\makeatletter
\title{{\large\textit{}}\\[0.5cm]{\Huge\color{gray}\textsc{\@docsubtitle}}}
\makeatother

\author{\textbf{Cordney Nash}  \and Micah Hillman  }
\date{September 30, 2024}
\footext{}



\begin{document}

\begin{titlepage}
\maketitle\vfill
\end{titlepage}
\newpage


\section*{Introduction}
{
This experiment explores the constructive diffraction of X-rays incidented on Lithium Fluoride (LiF) and Sodium chloride (NaCl) crystals and powdered crystalline samples. This experiment also aims to study the emission and absorption of X-rays by high atomic number (Z) metals. The resulting diffraction patterns show sharp maxima at specific angles, which are dependent on the wavelength of the X-rays and the crystal structure. A Geiger-Müller tube is used to detect measure the diffracted X-rays as a function of the scattering angle. From here different properties about the crystal can be determined, such as the distance between successive planes.
}

\begin{multicols}{2}

\section*{Theory \& Procedure}


\lipsum[4-8] 
    
\section*{Analysis \& Results}
    \lipsum[10-13]

\end{multicols}

\section*{Summary}
        \lipsum[14]
        \lipsum[15]


\end{document}